\documentclass{article}
\usepackage[textwidth=14cm]{geometry}
\usepackage[utf8]{inputenc}
\usepackage{MnSymbol}%
\usepackage{wasysym}%
\usepackage{amsfonts}
\usepackage{graphicx}
\usepackage{grffile}
\usepackage{amsmath}
\usepackage{csquotes}
\usepackage{listings}
\usepackage{subcaption}
\usepackage{wrapfig}
\usepackage{esvect}
\usepackage[section]{placeins}
\usepackage{minted}
\usepackage{hyperref}
\usepackage{adjustbox}
\graphicspath{ {/Users/anthonymaylath/Documents/NYU/High_Performance_Computing/HW/HW1} }
\DeclareGraphicsExtensions{.pdf,.png,.jpg}

\batchmode

\title{High Performance Computing - HW 5}
\author{Anthony Maylath}

\lstset{
numbers=left, 
numberstyle=\small, 
numbersep=8pt, 
frame = single, 
language=Pascal, 
framexleftmargin=15pt}

\begin{document}

\maketitle

\begin{center}

The Courant Institute for Mathematical Sciences, New York University \\ 

\end{center}

\DeclareGraphicsExtensions{.pdf,.png,.jpg}

\setcounter{MaxMatrixCols}{13}

\section{Question 1}

My implementation contains the function $timeRing()$ to capture the latency of message passing over integers and $timeRingAr()$ to capture the bandwidth over an integer array of size 500,000. Each implmentation adds the rank of the sending process for each send executed in the function. For $timeRingAr()$ the rank is added to every element in the array. Hence, the time to compute these increments are included in the latency and bandwidth figures.\\

To test the performance of my code, I ran it with a varying number of processes on crunchy5 and crunchy6. I ran the code with the following command:\\

mpirun -np [num processes] --oversubscribe 
-H crunchy5.cims.nyu.edu, crunchy6.cims.nyu.edu ./int\_ring -i [num iterations]\\

Table \ref{mpi1} shows the performance of my code over various number of processes and iterations. I ran my code on CIMS between 2 and 2:30pm on Tuesday. To better understand the results, the reader can cross check load on these machines during these times. Time seems to go up when the number of processes or number of iterations goes up. This makes sense as the number of processes increases the ring size. Latency and Bandwidth tend to increase with number of processes as messages need to communicate between more cores. Both Latency and Bandwidth tend to be smallest when number of iterations and number of processes are high. This occurs as the MPI overhead is small compared to the number of work in this instance.

\begin{table}[h!]
\centering
\begin{tabular}{ |c|c|c|c|c|c| }
\hline
Processes&	Iterations&	Time Primative&	Time Array&	Latency&	Bandwidth (GB/s)\\
\hline
2&	1&	0.000278&	0.036327&	0.000139&	0.041291\\
\hline
10&	1&	0.003824&	0.062924&	0.000382&	0.023838\\
\hline
30&	1&	0.049308&	0.248918&	0.001644&	0.006026\\
\hline
2&	100&	0.021569&	4.608056&	0.000108&	0.000326\\
\hline
10&	100&	0.136884&	7.396445&	0.000137&	0.000203\\
\hline
30&	100&	0.696036&	53.031526&	0.000232&	0.000028\\
\hline
2&	1000&	0.318983&	38.538305&	0.000159&	0.000039\\
\hline
10&	1000&	0.576416&	77.588472&	0.000058&	0.000019\\
\hline
\end{tabular}
 \caption{Performance on crunchy5 and crunchy6}
 \label{mpi1}
 \end{table}

\newpage

\section{Question 2}


\begin{table}[h!]
\centering
\begin{adjustbox}{width=\textwidth}
\begin{tabular}{ |c|c|c| }
\hline
Week&	Work&	Who\\
\hline
4/8-4/14 & Identify Datasets for clustering algorithm & Anthony Maylath and John Donaghy\\
 & Build datareader to read dataset & John\\
\hline
4/15-4/21 & Code Serial k-means and verify with sklearn python script& Anthony\\
 & Implement Initial MPI Parallel k-means & John\\
 & Debug Initial MPI implementation to match serial version.  & John and Anthony\\
 & Optimize algorithm to use fewer message passes & John and Anthony\\
 & Implement AVX logic to compute new clusters & John\\
 \hline 
 4/22-4/28 & Upgrade data parser to handle larger datasets & John\\
 & Set up scratch disk & John\\
 & Research MPI implementations for EM algorithm & Anthony\\
 \hline
 4/29-5/5 & Run scalability checks on big data. Check k-means performance & Anthony and John\\
 & (Time permitting) Implement serial EM algorithm. Verify with python & Anthony\\
 & Final optimizations on k-means MPI implementation & John and Anthony\\
 \hline
 5/6-5/12 & Start drafting report and slides & John and Anthony\\
 & Build MPI version of EM algorithm (time permitting) & John and Anthony\\
 & Go to office hours to get advise on how to improve things & John and/or Anthony\\
 & Finalize k-means implemenation & John and Anthony\\
 \hline
 5/13-5/22 & (Time permitting) Finalize EM algorithm and compare to k-means & John and Anthony\\
 & Run final tests on code base & John and Anthony\\
 & Finish report and slides & John and Anthony\\
 \hline
\end{tabular}
\end{adjustbox}
 \caption{MPI Clustering Algorithms}
 \label{mpi2}
 \end{table}


\end{document}